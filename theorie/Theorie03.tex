\documentclass[a4paper,12pt]{article}
\usepackage[left=2.5cm, right=2.5cm, top=2.5cm, bottom=2.5cm]{geometry}
\usepackage{siunitx}
\sisetup{locale=DE}
\usepackage{a4}	
\usepackage{epsfig}
\usepackage{float} 
\usepackage{ngerman}
\usepackage{amsmath,amssymb,amstext}
\usepackage[utf8]{inputenc}
\usepackage{graphicx}
\usepackage{setspace} %(\onehalfspacing \doublespacing \singlespacing)
\usepackage{eurosym} %Für € Symbol (\euro)
\usepackage{url}
\usepackage{hyperref}
\usepackage{enumitem}
%\pagestyle{empty} %Blendent Seitenzahlen aus
%Kopfzeile
\usepackage{fancyhdr}
\usepackage{color}
\usepackage{multicol}
\setlength{\parindent}{0em}

\newcommand{\floor}[1]{\left\lfloor#1\right\rfloor}
\newcommand{\ceil}[1]{\left\lceil#1\right\rceil}
\newcommand{\abs}[1]{\left|#1\right|}
\newcommand{\norm}[1]{\left|\left|#1\right|\right|}
\newcommand{\scal}[1]{\langle#1\rangle}

\newcommand{\dS}{\\ \\}

%Requires a math environment
\newcommand{\ex}{\hat{e}_x}
\newcommand{\ey}{\hat{e}_y}
\newcommand{\ez}{\hat{e}_z}
\newcommand{\et}{\hat{e}_\theta}
\newcommand{\er}{\hat{e}_r}
\newcommand{\R}{\mathbb{R}}
\newcommand{\N}{\mathbb{N}}
\newcommand{\p}[2]{\frac{\partial#1}{\partial#2}}

\newcommand{\wrongValue}[0]{\colorbox{red}{Falsche Werte}}
\newcommand{\todo}[0]{\colorbox{green}{Todo:}}

\newcommand{\vecValue}[1]{
	\begin{pmatrix}
		#1
	\end{pmatrix}
}

\pagestyle{fancy}
\fancyhf{}
\chead{Raketengleichung}
\headsep 1cm
\rfoot{\thepage} 

\begin{document}
	
\section*{Raketengleichung} %Ausgangsströmungsgeschwindigkeit

Die Raketengleichung beruht auf dem 3. Newtonsches Gesetz. Dies ist auch bekannt unter dem Namen des \textit{Wechselwirkungsprinzip} oder \textit{actio gleich reactio}.\\
Der Impuls der Rakete ist definiert durch:\\

$\begin{aligned}
	p_R = m\cdot dv
\end{aligned}$\\

Der Impuls, der durch das Gas erzeugt wird ist definiert durch:\\

$\begin{aligned}
	p_G = u\cdot dm
\end{aligned}$\\

Durch actio $=$ reactio gilt:\\

$\begin{aligned}
	                &&-\tilde{m}\cdot d\tilde{v}           &=&& u \cdot d\tilde{m}\\
	\Leftrightarrow &&-u^{-1}\cdot dv                      &=&& \tilde{m}^{-1} \cdot d\tilde{m}\\
	\Leftrightarrow &&-\int_{0}^{v} u^{-1}\cdot d\tilde{v} &=&& \int_{m}^{m_0} \tilde{m}^{-1} \cdot dm\\
	\Leftrightarrow && -\frac{v}{u}                        &=&& \ln(m_0) - \ln(m)\\
	\Leftrightarrow && -\frac{v}{u}                        &=&& \ln\left(\frac{m_0}{m}\right)\\
	\Leftrightarrow && v                                   &=&& -u\cdot \ln\left(\frac{m_0}{m}\right)\\
	\Rightarrow     && v(m)                                &=&& -u\cdot \ln\left(\frac{m_0}{m}\right)\\
	\Leftrightarrow && v(m)                                &=&& u\cdot \ln\left(\frac{m}{m_0}\right)
\end{aligned}$\\

Hier haben wir die Herleitung der Geschwindigkeit $v(m)$ der Raketengleichung in Abhängigkeit der Maße.\\

Möchte man die Geschwindigkeit der Rakete in Abhängigkeit der Zeit als $v(t)$, so ist die Maße $m$ von $t$ Abhängig. Daraus folgt dann die folgende Gleichung:\\

$\begin{aligned}
	v(t)=u\cdot \ln\left(\frac{m(t)}{m_0}\right)
\end{aligned}$\\

Wenn die Annahme getroffen wird, dass der Maße zur Zeit linear abnimmt also Treibstoffverbrauch ist konstant ist, gilt:\\

$\begin{aligned}
	\tilde{m}'(t) = k
\end{aligned}$\\

Daraus folgt dann.\\

$\begin{aligned}
	\int_{0}^{t}\tilde{m}'(\tilde{t})d\tilde{t} = \int_{0}^{t}kd\tilde{t} = k\cdot t
\end{aligned}$\\

Aus der Oberen Gleichung kann man die Gleichung für $m(t)$ bestimmen. Diese lautet:\\

$\begin{aligned}
	m(t) = m_0 - kt
\end{aligned}$\\

Mit der Gleichung $m(t)$ kann hat man nun die Gleichung $v(t)$ bestimmen. Diese lautet:\\

$\begin{aligned}
	v(t)=u\cdot \ln\left(\frac{m_0 - kt}{m_0}\right) = u\cdot \ln\left(1 - \frac{kt}{m_0}\right)
\end{aligned}$


\end{document}